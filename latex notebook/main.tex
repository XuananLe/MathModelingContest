\documentclass[12pt]{report}
\usepackage{subfigure}
\usepackage[utf8]{vietnam}
\usepackage[left=2.50cm, right=2.00cm, top=2.00cm, bottom=2.00cm]{geometry}
\usepackage{fancybox,graphicx}
\usepackage{mathrsfs} 
\usepackage{amsfonts}
\usepackage{longtable,array}
\usepackage{multirow}
\newlength\mylength
\newcolumntype{C}[1]{>{\centering\arraybackslash}p{#1}}
\usepackage[intlimits]{amsmath}
\usepackage{array}
\usepackage[unicode]{hyperref}
\usepackage{adjustbox}
% \usepackage{algorithm}
% \usepackage{algorithmicx}
\makeatletter
% \renewcommand{\ALG@name}{Thuật toán}
\makeatother
\usepackage{algpseudocode}
\usepackage{caption}
\usepackage{amsxtra,amssymb,latexsym,amscd,amsthm}
\usepackage{enumitem}
\usepackage{tikz}
\usepackage{makeidx}
\usepackage{float}
% \usetikzlibrary{shapes.geometric}
% \usetikzlibrary{positioning,automata}
% \usepackage{scrextend}
% \usepackage{longfbox}
%Môi trường Lời giải
%\newtheorem{theorem}{Định lý}[chapter]
%\newtheorem{definition}{Định nghĩa}[chapter]
%\newtheorem{example}{Ví dụ}[chapter]
%\newtheorem{lemma}[theorem]{Bổ đề}
%Tiêu đề
% \newtheorem{pro}{Bài toán}
% \newtheorem*{constr}{Ràng buộc}
% \newtheorem*{calfunc}{Các hàm được thực thi}
% \newtheorem*{Sol}{Giải thuật}
% \newtheorem*{Anal}{Phân tích giải thuật}

% \usepackage{fancyhdr}
% \pagestyle{fancy}
% \lhead{}
% \chead{}
% \rhead{Nguyên lý Hệ Điều Hành}
% \lfoot{}
% \cfoot{\thepage}
% \rfoot{}

\paperheight=24cm
\paperwidth=17cm
\textwidth=13truecm
\textheight=19,7truecm
\topmargin=0cm
\headsep=20truept
\headheight=12pt
\voffset=-0.75truecm
\hoffset=-0.75cm
\oddsidemargin=0.5cm
\evensidemargin=0.5cm
\footskip=30 pt
\pagenumbering {arabic}
\setcounter{page}{1}
\setcounter{chapter}{0}
\addtocontents{toc}{\protect\vspace{20pt}}

\usepackage{xcolor}
\usepackage{listings}
\usepackage{titlesec}

\titleformat{\chapter}[display]
{\normalfont\huge\bfseries}{\chaptertitlename\ \thechapter}{20pt}{\Huge}
\titlespacing*{\chapter}{0pt}{-50pt}{40pt}

% \renewcommand{\thechapter}{\Roman{chapter}}
% \renewcommand{\thesection}{\thechapter.\Roman{section}}
% \renewcommand{\thesubsection}{\thesection.\Roman{subsection}}

\definecolor{mGreen}{rgb}{0,0.6,0}
\definecolor{mGray}{rgb}{0.5,0.5,0.5}
\definecolor{mPurple}{rgb}{0.58,0,0.82}
\definecolor{backgroundColour}{rgb}{0.95,0.95,0.92}

\lstdefinestyle{CStyle}{
	backgroundcolor=\color{backgroundColour},
	commentstyle=\color{mGreen},
	keywordstyle=\color{magenta},
	numberstyle=\tiny\color{mGray},
	stringstyle=\color{mPurple},
	basicstyle=\footnotesize,
	breakatwhitespace=false,
	breaklines=true,
	captionpos=b,
	keepspaces=true,
	numbers=left,
	numbersep=5pt,
	showspaces=false,
	showstringspaces=false,
	showtabs=false,
	tabsize=2,
	language=C
}


% \renewcommand{\fboxrule}{0.3pt}

\begin{document}

\thispagestyle{empty}


\begin{titlepage}
	\begin{figure}[h]
		\centering
		\includegraphics[width=0.5\textwidth]{images/logotmh.png}
	\end{figure}

	\begin{center}
		\textbf{{\Huge CUỘC THI}} \\[10pt]
		\textbf{\LARGE TOÁN MÔ HÌNH 2023}\\[10pt]
		\textbf{\LARGE VÒNG 1}\\[10pt]
		\textbf{ 01 tháng 08  - 20 tháng 08, 2023}
		\\[3.0cm]

		% \begin{center}
		% 	\textbf{Tên đội thi}: APM \\
		% 	\textbf{Thành viên:}
		% 		&$-$ Nguyễn Thành Phát \\
		% 		$-$ Lê Xuân An \\
		% 		- Phạm Công Minh \\
		% 	Ngày thực hiện: (15/08/23)
		% \end{center}

		\begin{tabular}{r l}
			\textbf{Tên đội thi}: &APM \\
			\textbf{Thành viên:}
				&$-$ Nguyễn Thành Phát \\
				&$-$ Lê Xuân An \\
				&$-$ Phạm Công Minh
		\end{tabular}
	\end{center}

	\vspace{\fill}
\end{titlepage}

\newpage

\tableofcontents
% \renewcommand{\thechapter}{\arabic{chapter}}
% \renewcommand{\thesection}{\thechapter.\arabic{section}}
% \renewcommand{\thesubsection}{\thesection.\arabic{subsection}}
% \setcounter{chapter}{0}

\newpage

\chapter{Thông tin cần thiết về bài toán}
\section{Giới thiệu về tỉnh Quảng Bình} % (fold)

\subsection{Giới thiệu sơ bộ}
\begin{flushleft}
	Tỉnh Quảng Bình là một tỉnh nằm ở phía Bắc miền Trung Việt Nam, giáp biên giới với Lào. Tỉnh này có vị trí chiến lược trên tuyến đường biển Bắc - Nam của Việt Nam, và nổi tiếng với các danh lam thắng cảnh độc đáo, cùng với lịch sử và văn hóa đa dạng.
	\\[\baselineskip]

	\begin{figure}[h!]
		\centering
		\includegraphics[width = \textwidth]{images/BandoQB.png}
		\caption{Bản đồ hành chính tỉnh Quảng Bình}
	\end{figure}
\end{flushleft}

\newpage

\subsection{Đặc điểm khí hậu} % (fold)
\begin{flushleft}
	Quảng Bình thuộc vùng khí hậu nhiệt đới, chia thành hai mùa rõ rệt là mùa mưa và mùa khô. Dưới đây là mô tả sơ bộ về khí hậu của Quảng Bình:
	\begin{itemize}
		\item \textbf{Mùa mưa (mùa hè)}: Thường kéo dài từ tháng 9 đến tháng 3 năm sau. Trong khoảng thời gian này, tỉnh thường trải qua mưa nhiều và độ ẩm cao. Mùa mưa thường đem lại lượng mưa lớn, có thể gây ra lũ lụt và ngập úng tùy thuộc vào từng năm.

		\item \textbf{Mùa khô (mùa đông)}: Bắt đầu từ tháng 4 đến tháng 8. Trong mùa này, Quảng Bình thường trải qua ít mưa hơn và thời tiết khô ráo hơn. Nhiệt độ thường ổn định hơn so với mùa mưa.
	\end{itemize}

	Trong cả hai mùa, nhiệt độ ở Quảng Bình không có sự biến đổi lớn và thường duy trì ở mức ấm áp. Nhiệt độ trung bình hàng ngày dao động từ 22 đến 27 độ C, tùy thuộc vào mùa và thời điểm trong ngày. Mùa hè có thể có nhiệt độ cao hơn và cảm nhận nóng bức do tác động của ánh nắng mặt trời.
	\\[\baselineskip]

	Các biến đổi khí hậu và thời tiết trong Quảng Bình cũng ảnh hưởng đến việc trồng trọt và sản xuất nông nghiệp. Những biện pháp ứng phó với lũ lụt, ngập úng, và thiếu nước cần được đưa ra để đảm bảo sự ổn định trong sản xuất nông nghiệp và đời sống cộng đồng trong khu vực này.
\end{flushleft}

\chapter{Bài toán 1}
\section{Ý (a)} % (fold)

\begin{flushleft}
	Nông nghiệp nói chung và canh tác lúa có quan hệ qua lại và phức tạp đối với nhiều điều kiện, trong đó các yếu tố khí hậu là những nhân tố tác động mãnh mẽ nhất. Những ảnh hưởng của khí hậu, thời tiết đến sản xuất nông nghiệp được thể hiện qua đại lượng năng suất (cao hay thấp) và chất lượng nông sản (tốt hay xấu).
	\\[\baselineskip]

	Những điều kiện khí hậu thời tiết được xác định cho việc canh tác lúa trước hết là ánh sáng, nhiệt độ và nước. Đó là những yếu tố thiết yếu với sự sinh trưởng phát triển và cấu thành năng suất cây trồng và động vật. Chúng ta dễ dàng thấy rằng, để trồng được một cây lúa cần phải đảm bảo một lượng nhiệt (tổng nhiệt độ) nhất định, đồng thời phải có một lượng nước cần thiết trong tầng đất canh tác cho cả một thời kỳ sinh trưởng, phát triển. Bên cạnh đó, biến đổi khí hậu đóng vai trò rất quan trọng trong năng suất của cây lúa.
\end{flushleft}

\subsection{Ảnh hưởng của ánh sáng đến cây lúa} % (fold)
\label{sub:ảnh_hưởng_của_ánh_sáng_đến_cây_lúa}
\begin{flushleft}
	Ánh sáng có mối quan hệ quan trọng với sự phát triển của cây lúa và các loại cây khác. Dưới đây là một số điểm nổi bật về mối quan hệ này:

	\begin{itemize}
		 \item \textbf{Quá trình quang hợp}: Ánh sáng là yếu tố quan trọng nhất trong quá trình quang hợp, nơi cây lúa chuyển đổi ánh sáng mặt trời thành năng lượng thông qua việc sản xuất glucose từ nước và carbon dioxide. Glucose sau đó được dùng làm nhiên liệu cho các hoạt động của cây và giúp cây lúa phát triển.

		 \item \textbf{Chỉnh sửa sinh học}: Ánh sáng ảnh hưởng đến nhiều quá trình sinh học khác nhau trong cây lúa, chẳng hạn như sự mở rộng của lá, sự phát triển của rễ, và sự chín của bông lúa.

		 \item \textbf{Photoperiodism}: Một số giống lúa phản ứng với độ dài của ngày ánh sáng, gọi là photoperiodism. Điều này ảnh hưởng đến thời điểm cây ra hoa và thời gian thu hoạch. Sự hiểu biết về photoperiodism giúp nhà nghiên cứu và nông dân lựa chọn giống lúa phù hợp với điều kiện ánh sáng cụ thể của khu vực họ trồng.

		 \item \textbf{Nhiệt độ}: Ánh sáng mặt trời cũng là nguồn nhiệt độ quan trọng, ảnh hưởng đến sự phát triển và sinh trưởng của cây lúa.

		 \item \textbf{Kích thích sự phát triển}: Ánh sáng có khả năng kích thích sự phát triển của cây lúa thông qua việc ảnh hưởng đến các hoạt động hóa học và sinh lý.

		 \item \textbf{Quản lý sâu bệnh}: Ánh sáng cũng có thể ảnh hưởng đến sự xuất hiện và phát triển của một số loại sâu bệnh, ảnh hưởng đến năng suất và chất lượng của cây lúa.
	\end{itemize}

	Để đạt được năng suất tốt và chất lượng hạt lúa cao, việc quản lý và tối ưu hóa nguồn ánh sáng cho cây lúa là rất quan trọng.
	\\[\baselineskip]

	\begin{figure}[H]
		\centering
		\includegraphics[width = 0.6\textwidth]{images/sodoanhsang.png}
		\caption{Ảnh hưởng của ánh sáng đến với cường độ quang hợp của cây trồng}
		\label{fig:image}
	\end{figure}
\end{flushleft}
% subsection ảnh_hưởng_của_ánh_sáng_đến_cây_lúa (end)

\subsection{Ảnh hưởng của nhiệt độ đến cây lúa} % (fold)
\label{sub:ảnh_hưởng_của_nhiệt_độ_đến_cây_lúa}
\begin{flushleft}
	Nhiệt độ là một trong những yếu tố quan trọng ảnh hưởng đến sự sinh trưởng và phát triển của cây lúa. Dưới đây là một số cách mà nhiệt độ có thể ảnh hưởng đến lúa:

	\begin{itemize}
		\item \textbf{Nảy mầm}: Để hạt lúa nảy mầm tốt, nhiệt độ tối ưu thường nằm trong khoảng từ 25°C đến 35°C. Nếu nhiệt độ quá thấp hoặc quá cao, tỉ lệ nảy mầm có thể giảm.

		\item \textbf{Sự sinh trưởng của cây non}: Nhiệt độ quá cao hoặc quá thấp có thể làm giảm tốc độ sinh trưởng của cây non và giảm năng suất.

		\item \textbf{Quá trình hình thành bông}: Nhiệt độ quá cao vào thời điểm hình thành bông có thể làm giảm số lượng hạt trên mỗi bông và giảm chất lượng hạt.

		\item \textbf{Quá trình ra hoa và thụ phấn}: Nhiệt độ quá cao hoặc quá thấp có thể ảnh hưởng đến quá trình ra hoa và thụ phấn, làm giảm năng suất.

		\item \textbf{Chất lượng hạt}: Nhiệt độ ổn định và mát mẻ ở giai đoạn cuối cùng của sự phát triển hạt có thể giúp cải thiện chất lượng hạt.

		\item \textbf{Tốc độ chín}: Nhiệt độ cao có thể tăng tốc độ chín của lúa, nhưng cũng có thể làm giảm chất lượng hạt.

		\item \textbf{Bệnh và sâu hại}: Nhiệt độ cũng ảnh hưởng đến sự phát triển và sinh sản của nhiều loại bệnh và sâu hại.

		\item \textbf{Tích trữ nước}: Nhiệt độ cao làm gia tăng sự bay hơi nước từ mặt đất và từ lá cây, dẫn đến nhu cầu nước tăng lên.
	\end{itemize}

	Nói chung, việc quản lý và kiểm soát nhiệt độ trong quá trình trồng lúa rất quan trọng để đạt được năng suất và chất lượng tốt.
	\\[\baselineskip]

	\begin{figure}[H]
		\centering
		\caption{Liên hệ giữa nhiệt độ và năng suất lúa}
		\includegraphics[width = \textwidth]{images/sodonhietdo.png}
		\label{fig:image}
	\end{figure}

	Biểu đồ mô tả quan hệ giữa nhiệt độ và năng suất. Trong trường hợp không lai tạo để chịu nhiệt, việc tăng nhiệt độ trung bình vượt quá nhiệt độ tối ưu (${\blacklozenge}$) sẽ dẫn đến giảm năng suất trung bình và tăng biến động của năng suất, giả sử biến động nhiệt độ hàng năm vẫn không đổi.
\end{flushleft}
% subsection ảnh_hưởng_của_nhiệt_độ_đến_cây_lúa (end)

\newpage

\subsection{Ảnh hưởng của lượng nước đến cây lúa} % (fold)
\label{sub:ảnh_hưởng_của_lượng_nước_đến_cây_lúa}
\begin{flushleft}
	\textbf{Nhu cầu nước của lúa}: Nói chung nhu cầu nước của cây lúa lớn hơn so với một số cây trồng khác. Trước đây, ở nước ta cũng như một số nước trong khu vực, khi chưa có công trình thủy lợi thì hàng năm chỉ gieo cấy được một vụ vào mùa mưa. Nguồn nước mưa rất quan trọn, nó không chỉ cung cấp nước cho cây trồng sinh trưởng và phát triển mà còn làm thay đổi tiểu khí hậu trong ruộng lúa. Những cơn mưa nhiệt đới mang đến nguồn đạm từ khí trời và mang đến nguồn ôxi cho ruộng lúa. Trong điều kiện thủy lợi chưa hoàn chỉnh, hoặc hệ thống công trình xuống cấp hoặc hồ chứa không xả nước thì lượng mưa là một trong những yếu tố quyết định đến việc hình thành các vùng trồng lúa và các vụ lúa trong năm. Trong mùa mưa ẩm, lượng mưa cần thiết cho cây lúa trung bình là 6-7mm/ngày và 8-9mm/ngày trong mùa khô nếu không có các nguồn nước khác bổ sung. Nếu tính luôn lượng nước thấm và bốc hơi thì trung bình một tháng cây lúa cần khoảng 200mm nước mưa và trong cả vụ lúa cần 1000mm, chưa kể lượng nước cần thiết để gieo mạ.

	\begin{figure}[!htbp]
		\centering
		\includegraphics[width = \textwidth]{images/sodonuoc.png}
		\caption{Biểu đồ nước cho cây lúa}
		\label{fig:image}
	\end{figure}
\end{flushleft}
% subsection ảnh_hưởng_của_lượng_nước_đến_cây_lúa (end)

\newpage

\subsection{Ảnh hưởng của biến đổi khí hậu đến với năng suất lúa} % (fold)
\label{sub:ảnh_hưởng_của_biến_đổi_khí_hậu_đến_với_năng_suất_lúa}
\begin{flushleft}
	Biến đổi khí hậu đang và sẽ tiếp tục tác động mạnh mẽ đến năng suất lúa vì yếu tố này sẽ ảnh hưởng trực tiếp đến nhiệt độ, lượng nước mưa và chất lượng không khí.

	\begin{itemize}
		\item \textbf{Tăng nhiệt độ}: Lúa, đặc biệt là lúa nước, yêu cầu một nhiệt độ ổn định trong quá trình sinh trưởng. Tăng nhiệt độ có thể gây ảnh hưởng đến quá trình hình thành bông và làm giảm năng suất hạt. Nhiệt độ đêm tăng cao có thể giảm năng suất lúa do giảm sự tích luỹ chất dinh dưỡng trong quá trình photosynthesis.

		\item \textbf{Thay đổi mức lượng mưa}: Một số khu vực có thể trải qua giai đoạn hạn hán kéo dài, trong khi khu vực khác lại chịu cảnh lũ lụt thường xuyên hơn. Cả hai tình huống này đều ảnh hưởng tiêu cực đến năng suất lúa. Sự biến động không lường trước về mưa có thể gây ra khó khăn trong việc lên kế hoạch trồng trọt, đặc biệt là trong những khu vực phụ thuộc vào mùa mưa.

		\item \textbf{Tăng nồng độ $CO_2$}: Việc tăng nồng độ $CO_2$ có thể tăng cường quá trình quang hợp, giúp cây lúa phát triển nhanh hơn. Tuy nhiên, lợi ích này có thể bị giảm sút nếu không có đủ nước và dinh dưỡng. Các nghiên cứu đã chỉ ra rằng, dù $CO_2$ giúp tăng năng suất, nhưng chất lượng hạt lúa (như hàm lượng protein) có thể bị giảm.

		\item \textbf{Sự gia tăng của các sự kiện khí hậu cực đoan}: Sự gia tăng về số lần và mức độ của các sự kiện khí hậu cực đoan như bão, hạn hán và lũ lụt có thể gây ra tổn thất lớn cho vụ mùa.

		\item \textbf{Ảnh hưởng đến dịch hại và bệnh tật}: Biến đổi khí hậu có thể tạo điều kiện thuận lợi cho sự phát triển và lan rộng của các loại dịch hại và bệnh tật trên cây lúa, làm giảm năng suất và chất lượng sản phẩm.

		\item \textbf{Thay đổi về môi trường sinh thái}: Biến đổi khí hậu có thể làm thay đổi môi trường sinh thái xung quanh, ảnh hưởng đến độ mặn của nước trong khu vực trồng lúa ven biển hay thay đổi mức nước ngầm ở một số khu vực.
	\end{itemize}

	Những tác động trên đều yêu cầu các nhà nghiên cứu, nhà quản lý và người nông dân phải tìm kiếm giải pháp để thích nghi và giảm thiểu hậu quả của biến đổi khí hậu đối với ngành sản xuất lúa.
\end{flushleft}
% subsection ảnh_hưởng_của_biến_đổi_khí_hậu_đến_với_năng_suất_lúa (end)

\section{Ý (b)} % (fold)
\label{sec:ý_}

\begin{flushleft}
	Sau khi đã tìm hiểu từ các nguồn, đội đã tổng hợp được sản lượng lúa của Quảng Bình từ năm 2000 - 2022
	\\[\baselineskip]

	\begin{figure}[!htbp]
		\centering
		\includegraphics[width = \textwidth]{images/sanluonglua.png}
		\caption{Sản lượng lúa của Quảng Bình từ năm 2000 - 2022}
		\label{fig:image}
	\end{figure}
\end{flushleft}
% section ý_ (end)

\newpage

\section{Ý (c)} % (fold)
\label{sec:ý_}

\begin{flushleft}
	Sau khi đã tìm hiểu từ các nguồn, đội đã tổng hợp được các cơn bão đổ bộ vào khu vực miền Trung nói chung và Quảng Bình nói riêng trong giai đoạn 2000 - 2022.
	\\[\baselineskip]

	\begin{table}[!ht]
		% \centering
		\captionsetup{justification = centering}
		\caption{Các cơn bão đổ bộ vào khu vực miền Trung và Quảng Bình giai đoạn 2000 - 2022}
		\begin{adjustbox}{width = 1.2\linewidth, center}
		\begin{tabular}{|l|l|l|l|l|l|}
		\hline
			Thứ tự & Bão & Thời gian đổ bộ & Thời gian kết thúc & Cấp độ lúc đổ bộ & Đổ bộ \\ \hline
			1 & Wutip & 30/9/2013 & 1/10/2013 & 12 & Quảng Bình \\ \hline
			2 & Doksuri & 15/9/2017 & 16/9/2017 & 12, 13 & Quảng Bình \\ \hline
			3 & Sinlaku & 2/8/2020 & 5/8/2020 & 7 & Thanh Hóa \\ \hline
			4 & Noul & 18/9/2020 & 19/9/2020 & 8, 9 & Huế \\ \hline
			5 & Saudel & 26/10/2020 & 26/10/2020 & 6, 7 & Đã suy yếu thành ÁTNĐ$^{(*)}$ khi đổ bộ vào Quảng Bình \\ \hline
			6 & Molave & 28/10/2020 & 29/10/2020 & 13 & Quảng Ngãi \\ \hline
			7 & Vamco & 15/11/2020 & 17/11/2020 & 8.5 & Quảng Bình \\ \hline
			8 & Podul & 30/08/2019 & 31/9/2019 & 9 & Quảng Bình \\ \hline
			9 & Damrey & 4/11/2017 & 5/11/2017 & 12 & Khánh Hòa \\ \hline
			10 & Lekima & 3/10/2007 & 4/10/2007 & 12 & Quảng Bình \\ \hline
			11 & Bão số 5 (ÁTNĐ$^{(*)}$ 17W) & 23/9/2006 & 25/9/2006 & 8 & Quảng Bình \\ \hline
			12 & Sơn Tinh & 18/7/2018 & 22/7/2018 & 6.5 & Suy yếu thành ÁTNĐ$^{(*)}$ \\ \hline
			13 & Talas & 17/7/2017 & 17/7/2017 & 10 & Hà Tĩnh \\ \hline
			14 & Ketsana & 26/9/2009 & 30/9/2009 & 13 & Quảng Ngãi \\ \hline
			15 & Xangsane & 1/10/2006 & 2/10/2006 & 13 & Đà Nẵng \\ \hline
			16 & Usagi & 10/8/2001 & 11/8/2001 & 8 & Quảng Bình \\ \hline
			17 & Mekkhala & 28/9/2008 & 30/9/2008 & 9 & Quảng Bình \\ \hline
			18 & Wukong & 9/9/2000 & 10/9/2000 & 10 & Hà Tĩnh \\ \hline
			19 & Vicente & 18/9/2005 & 19/9/2005 & 10 & Hà Tĩnh \\ \hline
			20 & Mindulle & 24/8/2010 & 25/8/2010 & 10, 11 & Nghệ An \\ \hline
			21 & Rai & 11/9/2016 & 13/9/2016 & 8 & Quảng Nam \\ \hline
			22 & Linfa & 11/10/2020 & 11/10/2020 & 8 & Quảng Ngãi \\ \hline
			23 & Nari & 15/10/2013 & 16/10/2013 & 13 & Đà Nẵng, Huế \\ \hline
			24 & Podul & 14/11/2013 & 15/11/2013 & 8 & Quảng Ngãi \\ \hline
			25 & Áp thấp nhiệt đới 18W & 18/9/2013 & 21/9/2013 & 7 & Huế \\ \hline
			26 & Áp thấp nhiệt đới 23W & 9/10/2017 & 10/10/2017 & 7 & Hà Tĩnh \\ \hline
			27 & Ofel & 16/10/2020 & 16/10/2020 & 6 & Đà Nẵng \\ \hline
			28 & Áp thấp nhiệt đới Wilma (bão số 13) & 6/11/2013 & 8/11/2013 & 8 & Khánh Hòa \\ \hline
			29 & Aere (bão số 6) & 13/10/2016 & 17/10/2016 & 7 & Huế \\ \hline
			30 & Sonca & 25/7/2017 & 29/7/2017 & 8 & Quảng Trị \\ \hline
			31 & Nangka & 14/10/2020 & 14/10/2020 & 9 & Thanh Hóa \\ \hline
			32 & Cempaka & 23/7/2021 & 24/7/2021 & 6 & Quảng Ninh \\ \hline
			33 & Koguma & 11/6/2021 & 13/6/2021 & 9 & Thanh Hóa \\ \hline

		\multicolumn{2}{c}{\footnotesize Note: ÁTNĐ: Áp thấp nhiệt đới} \\ \hline
		\end{tabular}
		\end{adjustbox}
	\end{table}

	Giám khảo có thể xem file rõ hơn trong thư mục lời giải của team.
\end{flushleft}

\chapter{Bài toán 2} % (fold)
\label{cha:bài_toán_2}
afwuofb
% chapter bài_toán_2 (end)

\end{document}